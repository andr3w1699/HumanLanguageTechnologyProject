\documentclass{article}


% ready for submission
\usepackage[final]{hlt_project_template}
\usepackage{natbib}
\bibliographystyle{plainnat}



\usepackage[utf8]{inputenc} % allow utf-8 input
\usepackage[T1]{fontenc}    % use 8-bit T1 fonts
\usepackage{hyperref}       % hyperlinks
\usepackage{url}            % simple URL typesetting
\usepackage{booktabs}       % professional-quality tables
\usepackage{amsfonts}       % blackboard math symbols
\usepackage{nicefrac}       % compact symbols for 1/2, etc.
\usepackage{microtype}      % microtypography
\usepackage{xcolor}         % colors


\title{Your project title}


\author{%
  Name Surname\\
  \texttt{email address} \\
  \And
  Name2 Surname2\\
  \texttt{email address2}
  \And
  Name3 Surname3\\
  \texttt{email address3}
  \And
  Name4 Surname4\\
  \texttt{email address4}
  \And
  Name5 Surname5\\
  \texttt{email address5}
}


\begin{document}


\maketitle


\begin{abstract}
  A short summary of your project (including highlights on your results/insights).
  Note that: the main text is limited to \textbf{10} content pages, including all figures and tables. Additional pages containing references, and technical appendices do not count.

\end{abstract}


\section{Introduction}
Including the general context of your project; the motivations; an introduction to what are the contributions of your work; a short guide to the rest of the report.

Note that (sub-)sections titles can be changed to your own preference while retaining the meaning of the content.

\section{Background}
Related works.
Example reference \citep{lecun2015deep}.

\section{Method}
All relevant information on the method(s) you employed.

\section{Experimental analysis}
\subsection{Task(s)}
Describe the task or tasks used in the experiments.
\subsection{Experimental settings}
Describe all the relevant aspects of the experimental setup used in your experiment (e.g., how you performed model selection for fine-tuning of hyper-parameters, all details regarding the learning / fine-tuning of your model, etc.)
\subsection{Results} Provide results (figures and tables with mounerical results should go here). Provide insights and comments on the achieved results (also comparatively with literature). Possible ablation studies go here.

\begin{table}
  \caption{Example.}
  \label{sample-table}
  \centering
  \begin{tabular}{lll}
    \toprule
    \multicolumn{2}{c}{Part}                   \\
    \cmidrule(r){1-2}
    Name     & Description     & Size ($\mu$m) \\
    \midrule
    Dendrite & Input terminal  & $\sim$100     \\
    Axon     & Output terminal & $\sim$10      \\
    Soma     & Cell body       & up to $10^6$  \\
    \bottomrule
  \end{tabular}
\end{table}

\begin{figure}
  \centering
  \fbox{\rule[-.5cm]{0cm}{4cm} \rule[-.5cm]{4cm}{0cm}}
  \caption{Sample figure caption.}
\end{figure}


Example url link:
\url{http://mirrors.ctan.org/macros/latex/contrib/natbib/natnotes.pdf}

\section{Discussion}
Overall discussion on the results, relevant insights.
Frame your considerations in the current landscape of relevant research.

\section{Conclusions}
Draw conclusions and possibly delineate future works / possible improvements.

\bibliography{bib.bib}


\small






%%%%%%%%%%%%%%%%%%%%%%%%%%%%%%%%%%%%%%%%%%%%%%%%%%%%%%%%%%%%

\appendix

\section{Appendix / supplemental material}


Optionally include supplemental material (complete proofs, additional experiments and plots) in appendix.

\end{document}